\documentclass{article}
\usepackage{xcolor}
\usepackage{url}
\usepackage{hyperref}
\usepackage{graphicx}
\usepackage{soul}
\usepackage{enumerate}
\usepackage{fancyhdr}
\usepackage{listings}
\usepackage{amsmath}

% Define custom colors
\definecolor{bg}{rgb}{0.64, 0.64, 0.82}
\definecolor{frame}{rgb}{0.59, 0.47, 0.71}
\definecolor{keyword}{rgb}{0.63, 0.36, 0.94}
\definecolor{comment}{rgb}{0.44, 0.5, 0.56}
\definecolor{string}{rgb}{0.56, 0.27, 0.52}

% Set custom listings options
\lstset{
    backgroundcolor=\color{bg},
    frame=single,
    rulecolor=\color{frame},
    basicstyle=\ttfamily\small,
    keywordstyle=\color{keyword}\bfseries,
    commentstyle=\color{comment},
    stringstyle=\color{string},
    showstringspaces=false,
    breaklines=true,
    xleftmargin=2mm,
    xrightmargin=2mm
}

% Header
\pagestyle{fancy}
\fancyhf{}
\fancyhead[L]{MAD 102 - Intro to Prog - Fall 2024}
\fancyhead[R]{Instructor: Aishwarya Rajasekaran \thepage}

\title{Inheritance - MAD 102 Week 11 Notes}
\author{Hia Al Saleh}
\date{November 13th, 2024}

\begin{document}
\maketitle
\tableofcontents
\newpage

\section{Introduction to Inheritance}
Inheritance is a fundamental concept in object-oriented programming that allows a class to inherit attributes and methods from another class, promoting code reuse and modularity. The class that inherits is known as the \textbf{derived class}, while the class it inherits from is the \textbf{base class}.

\begin{itemize}
    \item \textbf{Benefits of Inheritance}:
        \begin{itemize}
            \item Reduces redundancy by reusing code across multiple classes.
            \item Allows for updates in the base class to automatically reflect in derived classes.
            \item Simplifies the code structure by promoting a hierarchical organization.
        \end{itemize}
\end{itemize}

\section{Class Hierarchy}
Inheritance creates a \textbf{class hierarchy}, which models relationships in the form of parent and child classes.
\begin{itemize}
    \item \textbf{Parent (Base) Class}: Provides attributes and methods to be inherited by child classes.
    \item \textbf{Child (Derived) Class}: Inherits attributes and methods from the parent class and can introduce its own methods.
\end{itemize}

For example:
\begin{lstlisting}[language=Python]
class Animal:
    def speak(self):
        print("Some generic animal sound")

class Dog(Animal):  # Dog is a derived class of Animal
    def speak(self):
        print("Woof Woof")
\end{lstlisting}

In this example, \texttt{Dog} inherits from \texttt{Animal} and overrides the \texttt{speak} method.

\section{Overriding Methods}
A derived class can redefine or \textbf{override} a method inherited from the base class. This enables the derived class to provide a specialized behavior.

\begin{lstlisting}[language=Python]
class Animal:
    def speak(self):
        print("Some generic animal sound")

class Cat(Animal):
    def speak(self):
        print("Meow")
        
animal = Animal()
animal.speak()  # Outputs: Some generic animal sound

cat = Cat()
cat.speak()     # Outputs: Meow
\end{lstlisting}

Here, the \texttt{Cat} class overrides the \texttt{speak} method from \texttt{Animal} to provide a custom implementation.

\section{Multiple Inheritance}
In Python, a class can inherit from multiple base classes, which is called \textbf{multiple inheritance}. This is useful when a class needs features from more than one base class.

\begin{lstlisting}[language=Python]
class Walker:
    def walk(self):
        print("Walking on two legs")

class Flyer:
    def fly(self):
        print("Flying with wings")

class Bird(Walker, Flyer):  # Bird inherits from both Walker and Flyer
    pass

sparrow = Bird()
sparrow.walk()  # Outputs: Walking on two legs
sparrow.fly()   # Outputs: Flying with wings
\end{lstlisting}

The \texttt{Bird} class inherits methods from both \texttt{Walker} and \texttt{Flyer}.

\section{Mixins}
\textbf{Mixins} are classes used to extend the functionality of a class by providing additional methods. They are not intended to be standalone and are usually used with multiple inheritance.

\begin{lstlisting}[language=Python]
class StuntMixin:
    def jump(self, distance):
        print(f"Jumped {distance} meters!")

class CarefulMixin:
    def caution(self):
        print("Proceeding with caution.")

class DirtBike(StuntMixin, CarefulMixin):
    pass

dirt_bike = DirtBike()
dirt_bike.jump(10)       # Outputs: Jumped 10 meters!
dirt_bike.caution()       # Outputs: Proceeding with caution.
\end{lstlisting}

The \texttt{DirtBike} class uses \texttt{StuntMixin} and \texttt{CarefulMixin} to gain additional capabilities.


\section{Extended Python Code Example}
Below is an extended example of inheritance and mixins, combining all of the above concepts into a more complex class structure.

\begin{lstlisting}[language=Python]
# Base class
class Vehicle:
    def __init__(self, current_speed=0):
        self.current_speed = current_speed
    
    def description(self):
        print(f"Traveling at {self.current_speed} km/h")
    
    def make_noise(self):
        print("Vroom Vroom")

# Derived class inheriting from Vehicle
class Bicycle(Vehicle):
    def __init__(self, has_basket=False):
        super().__init__(0)
        self.has_basket = has_basket
    
    def description(self):
        super().description()
        print("But in the bike lane!")
    
    def make_noise(self):
        print("Ring Ring")

# Mixins for additional functionality
class StuntMixin:
    def jump(self, distance):
        print(f"Jumped {distance} meters!")
    
    def skid(self, distance):
        print(f"Skidded {distance} meters!")

class CarefulMixin:
    def avoid_obstacles(self):
        print("Carefully avoiding obstacles.")

# Multiple inheritance with Bicycle and mixins
class DirtBike(Bicycle, StuntMixin, CarefulMixin):
    def __init__(self, has_basket=False):
        super().__init__(has_basket)
    
    def make_noise(self):
        print("Braaaap")

# Demonstration of class functionality
car = Vehicle()
car.description()
car.make_noise()

bike = Bicycle(has_basket=True)
bike.description()
bike.make_noise()

dirt_bike = DirtBike()
dirt_bike.jump(5)
dirt_bike.skid(2)
dirt_bike.avoid_obstacles()
dirt_bike.current_speed = 20
dirt_bike.description()
dirt_bike.make_noise()
\end{lstlisting}

\begin{itemize}
    \item \textbf{Vehicle}: The base class that provides a description and make\_noise methods.
    \item \textbf{Bicycle}: A derived class that inherits from Vehicle, overriding the \texttt{make\_noise} method and extending \texttt{description}.
    \item \textbf{StuntMixin} and \textbf{CarefulMixin}: Provide additional capabilities to the DirtBike class.
    \item \textbf{DirtBike}: Combines all features of Vehicle, Bicycle, and the mixins to create a versatile class with multiple behaviors.
\end{itemize}

\newpage 

\section{Class Exercise}
\begin{itemize}
    \item Create a program that will store a list of contacts
    \item There are three types of contacts
    \item A person with a name and phone number
    \item A student with a name, a phone number, and the school they attend
    \item An employee with a name, a phone number, and the place they work
    \item A student is a person and an employee is also a person.
    \end{itemize}
The program will ask what type of contact to enter, and will continually ask until the user says they are done. Then it will display all the contacts entered.
\begin{itemize}
          \item If the contact is a person – it will display their name and phone number
          \item If the contact is a student – it will display their name and phone number and then the
          school they attend
          \item If the contact is an employee – it will display their work place followed by their name and
          phone number 
\end{itemize}
\subsection{Code}
\begin{lstlisting}[language=python]
    # Person class
class Person:
    # Constructor      
    def __init__(self, name, phone):
        self.name = name
        self.phone = phone
        
    # String representation of the object    
    def description(self):
        return f"{self.name} {self.phone}"
 
# Student class inherits from Person class          
class Student(Person):
    # Constructor
    def __init__(self, name, phone, school):
        super().__init__(name, phone)
        self.school = school
    # String representation of the object    
    def description(self):
        return f"{self.name} {self.phone} {self.school}"

# Employee class inherits from Person class
class Employee(Person):
          # Constructor
          def __init__(self, name, phone, work):
                      super().__init__(name, phone)
                      self.work = work
          # String representation of the object            
          def description(self):
                    return f"{self.work} {self.name} {self.phone}"
# Main program
contacts = []

while True:
          contact_type = input("Enter type of contact (person, student, employee) or done: ").lower()
          if contact_type == "done":
                    break
          name = input("Enter name: ")
          phone = input("Enter phone number: ")
          if contact_type == "person":
                    contact = Person(name, phone)
          elif contact_type == "student":
                    school = input("Enter school: ")
                    contact = Student(name, phone, school)
          elif contact_type == "employee":
                    work = input("Enter work place: ")
                    contact = Employee(name, phone, work)
          contacts.append(contact)
          
for contact in contacts:
          print(contact)
\end{lstlisting}
\end{document}
