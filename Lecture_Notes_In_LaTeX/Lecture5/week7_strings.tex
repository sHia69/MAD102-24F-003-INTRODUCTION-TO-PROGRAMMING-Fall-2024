\documentclass{article}
\usepackage{amsmath}
\usepackage{graphicx}
\usepackage{hyperref}
\usepackage{fancyhdr}
\usepackage{xcolor}
\usepackage{listings}

% Define custom colors
\definecolor{bg}{rgb}{0.64, 0.64, 0.82}
\definecolor{frame}{rgb}{0.59, 0.47, 0.71}
\definecolor{keyword}{rgb}{0.63, 0.36, 0.94}
\definecolor{comment}{rgb}{0.44, 0.5, 0.56}
\definecolor{string}{rgb}{0.56, 0.27, 0.52}

% Set custom listings options
\lstset{
    backgroundcolor=\color{bg},
    frame=single,
    rulecolor=\color{frame},
    basicstyle=\ttfamily\small,
    keywordstyle=\color{keyword}\bfseries,
    commentstyle=\color{comment},
    stringstyle=\color{string},
    showstringspaces=false,
    breaklines=true,
    xleftmargin=2mm,
    xrightmargin=2mm
}

% Header
\pagestyle{fancy}
\fancyhf{}
\fancyhead[L]{MAD 102 - Intro to Prog - Fall 2024}
\fancyhead[R]{Instructor: Aishwarya Rajasekaran \thepage}
\title{Python Strings - MAD 102 Week 7 Notes}
\author{Hia Al Saleh}
\date{October 17th, 2024}

\begin{document}
\maketitle
\tableofcontents
\newpage 

\section{Introduction to Strings in Python}
Strings are one of the most commonly used data types in Python. They are sequences of characters enclosed in either single or double quotes.

\begin{lstlisting}[language=Python]
# Initialize a string
string = "Hello World!"
print(string[0:4])  # Outputs 'Hell'
\end{lstlisting}

We can use indexing and slicing to extract parts of a string. Python supports both positive and negative indexing.

\subsection{Indexing and Slicing}
Indexing starts from 0, and negative indexing begins from -1 (last character).
\begin{lstlisting}[language=Python]
# Positive indexing
string[1]  # Outputs 'e'

# Negative indexing
string[-1]  # Outputs '!'
\end{lstlisting}

\subsection{String Length}
The length of a string can be determined using the \texttt{len()} function:
\begin{lstlisting}[language=Python]
len(string)  # Outputs 12
\end{lstlisting}

\section{Slicing Strings}
Slicing allows us to extract a range of characters from a string:
\begin{lstlisting}[language=Python]
word = 'batman'
print(word[3:5])  # Outputs 'ma'
print(word[-3:])  # Outputs 'man'
\end{lstlisting}

The start or end of the slice can be omitted:
\begin{lstlisting}[language=Python]
print(word[3:])   # Outputs 'man'
print(word[:3])   # Outputs 'bat'
\end{lstlisting}

\section{String Formatting}
Python provides several methods for formatting strings:
\subsection{Field Width and Alignment}
We can use formatted string literals (f-strings) to specify field width and alignment:
\begin{lstlisting}[language=Python]
print(f'{"Student Name":20}{"Marks":^10}')
# Output:
# Student Name          Marks   
# ------------------------------
# Malcola Reynolds        60    
\end{lstlisting}

\subsection{Fill Characters}
Padding characters can be specified in the format:
\begin{lstlisting}[language=Python]
print(f'{10:0>6}')  # Outputs '000010'
\end{lstlisting}

\section{String Methods}
Python strings support a variety of useful methods, including:
\subsection{replace()}
The \texttt{replace()} method replaces parts of a string with another substring:
\begin{lstlisting}[language=Python]
title = 'The new adventures of Indiana Jones'
print(title.replace('new', 'continuing'))
# Outputs: 'The continuing adventures of Indiana Jones'
\end{lstlisting}

\subsection{find()}
The \texttt{find()} method returns the index of the first occurrence of a substring:
\begin{lstlisting}[language=Python]
phrase = 'This is very, very, very long'
print(phrase.find('very'))  # Outputs: 8
\end{lstlisting}

\section{Comparing Strings}
Python supports comparison of strings using relational and equality operators:
\begin{lstlisting}[language=Python]
print('a' > 'A')  # Outputs: True
print('bat' > 'ball')  # Outputs: True
\end{lstlisting}

\subsection{Membership and Identity Operators}
We can check for substring membership or identity:
\begin{lstlisting}[language=Python]
print('bat' in 'batman')  # Outputs: True
print(string1 is string2)  # Checks if both variables refer to the same object
\end{lstlisting}

\section{Looping Through Strings}
We can iterate through strings using a \texttt{for} loop:
\begin{lstlisting}[language=Python]
word = 'batman'
for char in word:
    print(char)
# Outputs:
# b
# a
# t
# m
# a
# n
\end{lstlisting}

\section{String Validation Methods}
Python provides several methods for validating string content:
\begin{lstlisting}[language=Python]
print('abc123'.isalnum())  # Returns True if all characters are alphanumeric
print('123'.isdigit())  # Returns True if all characters are digits
print('abc'.islower())  # Returns True if all characters are lowercase
print('   '.isspace())  # Returns True if all characters are whitespace
\end{lstlisting}

\section{String Manipulation Methods}
Common string manipulation methods include:
\begin{itemize}
    \item \texttt{capitalize()} – Capitalizes the first character.
    \item \texttt{lower()} – Converts all characters to lowercase.
    \item \texttt{upper()} – Converts all characters to uppercase.
    \item \texttt{strip()} – Removes leading and trailing spaces.
    \item \texttt{title()} – Capitalizes the first letter of each word.
\end{itemize}
\begin{lstlisting}[language=Python]
phrase = " frozen is my FAVOURITE movie!!!   "
print(phrase.capitalize())  # Outputs: 'Frozen is my favourite movie!!!'
print(phrase.strip())  # Outputs: 'frozen is my FAVOURITE movie!!!'
\end{lstlisting}

\section{String Splitting and Joining}
The \texttt{split()} method splits a string into a list of substrings based on a separator:
\begin{lstlisting}[language=Python]
phrase = "I love to watch Frozen, Despicable Me, Free Birds"
print(phrase.split(','))  # Outputs: ['I love to watch Frozen', ' Despicable Me', ' Free Birds ']
\end{lstlisting}

The \texttt{join()} method joins a list of strings with a specified separator:
\begin{lstlisting}[language=Python]
list = ['https', 'www', 'google', 'com']
print('/'.join(list))  # Outputs: 'https/www/google/com'
\end{lstlisting}

\section{Password Validation Challenge}
A sample challenge involves validating a password based on certain criteria:
\begin{lstlisting}[language=Python]
email = input("Enter Email ID: ").lower().strip()
password = input("Enter Password: ")

if (len(password) >= 8 and 
    any(char.islower() for char in password) and
    any(char.isupper() for char in password) and
    any(char.isdigit() for char in password)):
    print("Password is valid")
else:
    print("Password is invalid")
\end{lstlisting}

This program ensures the password contains at least 8 characters, one lowercase, one uppercase, and one digit.

\end{document}
