\documentclass{article}
\usepackage{xcolor}
\usepackage{url}
\usepackage{hyperref}
\usepackage{graphicx}
\usepackage{soul}
\usepackage{enumerate}
\usepackage{fancyhdr}
\usepackage{listings}

% Define custom colors
\definecolor{bg}{rgb}{0.64, 0.64, 0.82}
\definecolor{frame}{rgb}{0.59, 0.47, 0.71}
\definecolor{keyword}{rgb}{0.63, 0.36, 0.94}
\definecolor{comment}{rgb}{0.44, 0.5, 0.56}
\definecolor{string}{rgb}{0.56, 0.27, 0.52}

% Set custom listings options
\lstset{
    backgroundcolor=\color{bg},
    frame=single,
    rulecolor=\color{frame},
    basicstyle=\ttfamily\small,
    keywordstyle=\color{keyword}\bfseries,
    commentstyle=\color{comment},
    stringstyle=\color{string},
    showstringspaces=false,
    breaklines=true,
    xleftmargin=2mm,
    xrightmargin=2mm
}
% Header
\pagestyle{fancy}
\fancyhf{}
\fancyhead[L]{MAD 102 - Intro to Prog - Fall 2024}
\fancyhead[R]{Instructor: Aishwarya Rajasekaran \thepage}

\title{Controlling Program Flow - MAD 102 Week 3 Notes}
\author{Hia Al Saleh}
\date{September 18th, 2024}
\begin{document}
\maketitle
\tableofcontents 
\newpage

\section{The Selection Structure}
\begin{itemize}
    \item A selection structure evaluates a condition, which is an expression that’s true or false.
    \item This allows you to specify different courses of action based on the evaluation:
    \begin{itemize}
        \item Do one thing if true.
        \item Do something else if not (false).
    \end{itemize}
    \item A branch is a sequence of statements that are only executed if a specific condition is met. This branch may never get executed in your program.
\end{itemize}

\section{Boolean Expression}
\begin{itemize}
    \item A selection structure depends on a condition.
    \item This is an expression describing the relationship between two values that’s evaluated when it appears in program code.
    \item A Boolean expression evaluates to true or false.
    \item Named after \href{https://en.wikipedia.org/wiki/George_Boole}{George Boole}, who developed an extensive system of logic based on true and false conditions and their consequences.
\end{itemize}

\section{Booleans and Equality Operator}
\begin{itemize}
    \item Booleans are used to compare values.
    \begin{itemize}
        \item Are you old enough to drive?
        \item Is the correct username entered?
        \item Did I successfully retrieve the information from the server?
    \end{itemize}
    \item To see if an answer is equivalent to an expected value, use the equality operator \texttt{==}.
    \item This returns a value of either true or false.
    \item \textbf{NOTE:} Many students confuse the mathematical operator \texttt{=} with the equality operator \texttt{==}. Remember that \texttt{=} is the assignment operator in coding.
\end{itemize}

\section{Relational Operators}
Common relational operators include:
    \begin{itemize}
        \item \texttt{\hl{==}}: Checks if two values are equal.
        \item \texttt{\hl{<}}: Less than.
        \item \texttt{\hl{>}}: Greater than.
        \item \texttt{\hl{<=}}: Less than or equal to.
        \item \texttt{\hl{>=}}: Greater than or equal to.
        \item \texttt{\hl{!=}}: Not equal to.
    \end{itemize}
The \texttt{==} operator is a common relational operator. Here are some additional relational operators used in programming:

\begin{itemize}
        \item \texttt{age < 60}: Checks if the age is less than 60.
        \item \texttt{hours > 40}: Checks if the hours are greater than 40.
        \item \texttt{region == "Ontario"}: Checks if the region is equal to Ontario.
        \item \texttt{status != "denied"}: Checks if the status is NOT denied.
        \item \texttt{quantity <= 10}: Checks if the quantity is 10 and under (includes 10).
        \item \texttt{grade >= 90}: Checks if the grade is greater than or equal to 90.
    \end{itemize}
\textbf{NOTE:} The relational operator for checking if two values are equal is \texttt{==}. A single equals sign \texttt{=} is the assignment operator.
\section{Controlling Program Flow}
\begin{itemize}
    \item The most common way of controlling a program's flow is to determine what a program will do based on decisions.
    \item These decisions can be simple or complex.
    \item The decision process is facilitated using an \texttt{if} statement.
\end{itemize}

\subsection{If Statement}
\begin{itemize}
    \item The simplest selection structure is one in which an action is taken if a condition evaluates to true, but no action is taken if the condition evaluates to false.
    \item This is a single-outcome section.
\end{itemize}

\begin{lstlisting}[language=Python]
# Python syntax
if condition:
    # action if condition is true
else:
    # action if condition is false
\end{lstlisting}
\textbf{NOTE:} An indentation is a tab – it is not a series of spaces.

\section{Flow Charting}
\begin{itemize}
    \item The diamond shape is the standard symbol for flow charting. It represents decision points with single outcomes.
    \item True Branch and False Branch lead to further actions based on evaluation.
\end{itemize}
\includegraphics[width=\linewidth]{Lecture_Notes_In_LaTex/imgs/flowchart.png}
\section{Dual Outcome}
\begin{itemize}
    \item A dual outcome is a selection statement where you perform one set of instructions if a condition evaluates to true, otherwise perform a second set of instructions.
    \item Dual outcomes work under the principles of Boolean logic – something is either true or false – if true, do this; otherwise, do the other step.
    \item Dual outcomes use the keywords \texttt{if} and \texttt{else}.
\end{itemize}

\begin{lstlisting}[language=Python]
if condition:
    # action if condition is true
else:
    # action if condition is false
\end{lstlisting}

\section{Multiple Outcomes}
\begin{itemize}
    \item When there are more than a single or dual outcomes, use \texttt{if}, \texttt{elif}, and \texttt{else} to represent multiple conditions.
    \item The first condition is marked with \texttt{if}.
    \item The end condition is marked with \texttt{else}.
    \item All other conditions are marked with \texttt{elif}.
\end{itemize}

\begin{lstlisting}[language=Python]
grade = 55

if grade >= 80:
    print("You have received an A")
elif grade >= 70:
    print("You have received a B")
elif grade >= 60:
    print("You have received a C")
elif grade >= 50:
    print("You have received a D")
else:
    print("You have received an F")
\end{lstlisting}

\section{Detecting Ranges}
\begin{itemize}
    \item The order of your conditional statements allows you to check if a value is within a specified range.
    \item Each expression indicates the upper range.
    \item If you fall to the next condition and that evaluates to true, you must be within the specified range.
\end{itemize}

\section{Describing Complex Conditions}
\begin{itemize}
    \item Often, two or more conditions are involved in a decision. Describe the relationship between the conditions.
    \item Example:
    \begin{itemize}
        \item A student makes the dean's list for taking 12 credit hours AND having a grade point average of at least 3.5.
        \item A movie theater offers a discount to anyone who is under 6 years old OR over 65 years old.
        \item An employee gets a bonus vacation day for meeting a sales quota AND not being absent for a three-month period.
    \end{itemize}
\end{itemize}

\section{Logical Operators: \texttt{and}, \texttt{or}, and \texttt{not}}
\begin{itemize}
    \item Logical operators are used when evaluating two or more conditions.
    \item A complex condition occurs when two or more conditions must be evaluated for an action to take place.
    \item Example:
    \begin{itemize}
        \item An employee is eligible for a discount on store items after working two months AND having a perfect attendance record.
    \end{itemize}
\end{itemize}

\begin{itemize}
    \item Conditions are joined with:
    \begin{itemize}
        \item \texttt{and} - both conditions must be true.
        \item \texttt{or} - at least one condition must be true.
        \item \texttt{not} - negates the condition; true if the value is false.
    \end{itemize}
\end{itemize}

\section{Truth Tables}
\begin{itemize}
    \item Truth tables help sort out complex logical situations and make coding easier.
    \item A truth table expresses the results of combinations of conditions.
\end{itemize}

\section{Decision Tables}
\begin{itemize}
    \item Decision tables are used for problems with multiple outcomes.
    \item They state all relevant conditions, true and false combinations of these conditions, and outcomes associated with each combination.
    \item The number of combinations is \(2^n\), where \(n\) is the number of conditions.
\end{itemize}

\begin{itemize}
    \item Example:
    \begin{itemize}
        \item A bank loan decision based on income, credit score, and employment duration.
    \end{itemize}
\end{itemize}

\section{Binary Trees}
\begin{itemize}
    \item Binary trees trace all combinations by splitting conditions into true and false paths.
    \item Paths lead to the next condition. Irrelevant conditions don’t split into true and false paths.
\end{itemize}

\section{Resulting Code}
\begin{itemize}
    \item The decision table can be translated into code.
\end{itemize}
\begin{lstlisting}[language=python]
# Resilting Code
student = ["Mal", "Jayne", "Washburn", "Zoe"]
name = input("Please enter a name:")

if name in student:
    print("Welcone - you are registered")
else:
    print("Welcome - you are not on our class list")        
\end{lstlisting}

\section{Nested Conditional Statements}
\begin{itemize}
    \item A branch’s statement can hold additional \texttt{if-else} statements, known as nested statements.
    \item Indentation is important.
\end{itemize}

\section{Membership Operators}
\begin{itemize}
    \item Membership operators return a Boolean value (true or false).
    \item They determine if a specified value is found in a container type (using \texttt{in}) or not (using \texttt{not in}).
\end{itemize}
\begin{lstlisting}[language=python]
# Membership Operator
firstName = "Edgar"
lastName = "Smith"

anotherName = firstName

if firstName is anotherName:
    print("Same object")
else:
    print("Different object")     
\end{lstlisting}
\section{Identity Operator}
\begin{itemize}
    \item The identity operator (\texttt{is}) checks if two operands are bound to a single object.
    \item \texttt{is not} is the inverse.
    \item They do not compare values but check if two variables share the same memory address.
\end{itemize}

\section{Ternary Operation}
\begin{itemize}
    \item A ternary operation is a conditional expression with three operands.
    \item Also known as a conditional expression.
    \item Difficult to read; should only be used for simple assignments.
\end{itemize}

\begin{lstlisting}[language=Python]
# Ternary Operation
grade = 50
result = "You have passed" if grade >= 50 else "You have failed"
print(result)
\end{lstlisting}

\end{document}
