\documentclass{article}
\usepackage{xcolor}
\usepackage{url}
\usepackage{hyperref}
\usepackage{graphicx}
\usepackage{soul}
\usepackage{enumerate}
\usepackage{fancyhdr}
\usepackage{listings}
\usepackage{amsmath}

% Define custom colors
\definecolor{bg}{rgb}{0.64, 0.64, 0.82}
\definecolor{frame}{rgb}{0.59, 0.47, 0.71}
\definecolor{keyword}{rgb}{0.63, 0.36, 0.94}
\definecolor{comment}{rgb}{0.44, 0.5, 0.56}
\definecolor{string}{rgb}{0.56, 0.27, 0.52}

% Set custom listings options
\lstset{
    backgroundcolor=\color{bg},
    frame=single,
    rulecolor=\color{frame},
    basicstyle=\ttfamily\small,
    keywordstyle=\color{keyword}\bfseries,
    commentstyle=\color{comment},
    stringstyle=\color{string},
    showstringspaces=false,
    breaklines=true,
    xleftmargin=2mm,
    xrightmargin=2mm
}

% Header
\pagestyle{fancy}
\fancyhf{}
\fancyhead[L]{MAD 102 - Intro to Prog - Fall 2024}
\fancyhead[R]{Instructor: Aishwarya Rajasekaran \thepage}

\title{Exceptions - MAD 102 Week 12 Notes}
\author{Hia Al Saleh}
\date{November 20th, 2024}

\begin{document}
\maketitle
\tableofcontents
\newpage

\section{Handling Errors}
\begin{itemize}
    \item When the user enters incorrect information – how do you handle it?
    \item How do you detect that the information is incorrect and what course of action do you take to ensure that your program continues to run?
    \item Create error-checking code to detect and handle errors while the program is executing.
\end{itemize}

\section{Exception Handling}
\begin{itemize}
    \item Special constructs known as \textbf{exception-handling} handle exceptional conditions.
    \item Example: User inputs \texttt{'dog'} instead of a number.
\end{itemize}

\section{Try / Except}
\begin{itemize}
    \item If there's a possibility that your code may produce an exception, place the code in a \texttt{try} block.
    \item Use the keyword \texttt{try} to mark the block.
    \item The solution for handling the exception is placed in an \texttt{except} block.
    \item Use the keyword \texttt{except} to mark the block.
\end{itemize}

\section{Try / Except Example}
\begin{itemize}
    \item If no errors occur in the \texttt{try} block (no exceptions), the code progresses as normal.
    \item If an error occurs (an exception is thrown), the code in the \texttt{except} block is executed.
    \item Any code in the \texttt{try} block is skipped if an error occurs.
\end{itemize}

\section{Exception Errors}
\begin{itemize}
    \item \textbf{ValueError} – Raised when an operation or function receives an argument with the right type but inappropriate value.
    \item \textbf{KeyError} – Raised when a mapping (dictionary) key is not found in the set of existing keys.
    \item \textbf{IndexError} – Raised when a sequence subscript is out of range.
    \item \textbf{NameError} – Raised when an identifier is not found.
    \item \textbf{ZeroDivisionError} – Raised when the second argument of a division or modulo operation is zero.
\end{itemize}

More built-in exception errors can be found in the Python documentation: \url{https://docs.python.org/3/library/exceptions.html}.

\section{Multiple Exception Handlers}
\begin{itemize}
    \item Multiple exceptions could be generated depending on your program.
    \item Use multiple \texttt{except} blocks to handle various types of exceptions.
    \item Each \texttt{except} block should specify the type of exception it handles.
\end{itemize}

Example:
\begin{lstlisting}[language=Python]
except (ValueError, TypeError):
    # Code to execute
\end{lstlisting}

\section{Raising Exceptions}
\begin{itemize}
    \item Exception handling constructs can be used to raise (throw) errors.
    \item The keyword \texttt{raise} is used, followed by the type of error to raise.
    \item A string argument is provided to the exception error that details the issue.
\end{itemize}

\section{Exceptions with Functions}
\begin{itemize}
    \item Exceptions can be part of functions, allowing for better program clarity.
\end{itemize}

\section{\texttt{finally} Clause}
\begin{itemize}
    \item The \texttt{finally} clause of a \texttt{try} statement specifies clean-up actions.
    \item These actions are always executed, regardless of whether an exception is raised.
\end{itemize}

\section{Creating Custom Error Types}
\begin{itemize}
    \item Custom exception types can be created using classes.
    \item Define a class with an initializer and use it as a normal exception.
\end{itemize}

\end{document}
