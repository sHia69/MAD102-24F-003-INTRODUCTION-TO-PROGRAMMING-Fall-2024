\documentclass{article}
\usepackage{xcolor}
\usepackage{url}
\usepackage{hyperref}
\usepackage{graphicx}
\usepackage{soul}
\usepackage{enumerate}
\usepackage{fancyhdr}
\usepackage{listings}
\usepackage{amsmath}

% Define custom colors
\definecolor{bg}{rgb}{0.64, 0.64, 0.82}
\definecolor{frame}{rgb}{0.59, 0.47, 0.71}
\definecolor{keyword}{rgb}{0.63, 0.36, 0.94}
\definecolor{comment}{rgb}{0.44, 0.5, 0.56}
\definecolor{string}{rgb}{0.56, 0.27, 0.52}

% Set custom listings options
\lstset{
    backgroundcolor=\color{bg},
    frame=single,
    rulecolor=\color{frame},
    basicstyle=\ttfamily\small,
    keywordstyle=\color{keyword}\bfseries,
    commentstyle=\color{comment},
    stringstyle=\color{string},
    showstringspaces=false,
    breaklines=true,
    xleftmargin=2mm,
    xrightmargin=2mm
}

% Header
\pagestyle{fancy}
\fancyhf{}
\fancyhead[L]{MAD 102 - Intro to Prog - Fall 2024}
\fancyhead[R]{Instructor: Aishwarya Rajasekaran \thepage}

\title{Lists and Dictionaries - MAD 102 Week 8 Notes}
\author{Hia Al Saleh}
\date{October 23rd, 2024}

\begin{document}
\maketitle
\tableofcontents
\newpage

\section{Nested Lists}
Lists can contain other lists, known as nested lists. For example:

\begin{lstlisting}[language=python]
crews = [['Mal', 'Washburne', 'Zoe'], ['Han', 'Chewie'], ['Kirk', 'Spock', 'McCoy']]
print(crews[1][1])  # Outputs 'Chewie'
\end{lstlisting}
Can use a for loop, with nested for loop, to iterate over the entire contents
\begin{lstlisting}[language=python]
crews = [['Mal', 'Washburne', 'Zoe'], ['Han', 'Chewie'], ['Kirk', 'Spock', 'McCoy']]
for position, crew in enumerate(crews):
        print('=' * 20)
        print(f'Crew #{position + 1}')
        print('=' * 20)
        for member in crew:
              print(member)
\end{lstlisting}

\section{List Slicing}
Slicing can be used to extract parts of a list. Syntax:
\[
\text{list}[ \text{startIndex} : \text{endIndex} ]
\]
Example:
\begin{lstlisting}[language=python]
numbers = [1, 2, 3, 4, 5]
sliced = numbers[:3]  # [1, 2, 3]
\end{lstlisting}

\section{List Comprehensions}
List comprehensions allow for modifying elements in a list:
\begin{lstlisting}[language=python]
numbers = [1, 2, 3, 4, 5]
new_list = [x * 2 for x in numbers]  # [2, 4, 6, 8, 10]
\end{lstlisting}

\section{Dictionaries}
Dictionaries store key-value pairs:
\begin{lstlisting}[language=python]
fruit_prices = {'apple': 2, 'banana': 1, 'orange': 3}
print(fruit_prices['apple'])  # Outputs 2
\end{lstlisting}

\section{Procedural Programming}
Up to this point, we've focused on procedural programming, executing steps in sequence with conditional statements and loops.

\section{Object Oriented Programming}
Object-Oriented Programming (OOP) is a paradigm that focuses on \textbf{objects}, containing both attributes and behaviors. Examples of objects include a \textit{person}, a \textit{car}, and a \textit{bank account}.

\subsection{Attributes and Behaviors}
Attributes represent the data stored in an object, while behaviors are actions that an object can perform. For example, a dog can \textit{bark}, \textit{run}, and \textit{eat}.

\subsection{Classes and Objects}
A \textbf{class} is a blueprint for creating objects, containing attributes and methods. Here's an example:
\begin{lstlisting}[language=python]
class Dog:
    def __init__(self, name):
        self.name = name
    def bark(self):
        print(f"{self.name} says Woof!")
\end{lstlisting}
\newpage 
\section{Class Exercises }
\subsection{Favorite Animals}
\textbf{Task:}
\begin{itemize}
    \item Ask the user for a name and their favorite animal.
    \item Keep asking until they quit.
    \item Display the list of names entered.
    \item Display the list of animals entered.
    \item Display names and their favorite animal.
\end{itemize}

\subsubsection{Python Code}
The following Python code implements the \textit{Favorite Animals} task.

\begin{lstlisting}[language=python]
def favorite_animals():
    names = []
    animals = []

    while True:
        name = input("Enter your name (or 'quit' to stop): ")
        if name.lower() == 'quit':
            break
        animal = input(f"What's {name}'s favorite animal?: ")
        names.append(name)
        animals.append(animal)

    print("\nList of Names:", names)
    print("List of Favorite Animals:", animals)
    print("\nNames and their favorite animals:")
    for i in range(len(names)):
        print(f"{names[i]}'s favorite animal is {animals[i]}")
favorite_animals()        
\end{lstlisting}
\newpage
\subsection{Mean - Median - Mode}
\textbf{Task:}
\begin{itemize}
    \item Ask the user to enter a series of numbers.
    \item Calculate the mean, median, and mode of the numbers.
    \item Use separate functions for each calculation.
\end{itemize}

\subsubsection{Python Code}
The following Python code implements the \textit{Mean, Median, and Mode} task.

\begin{lstlisting}[language=python]
import statistics

def calculate_mean_median_mode():
    numbers = list(map(float, input("Enter numbers separated by space: ").split()))

    mean = statistics.mean(numbers)
    median = statistics.median(numbers)
    try:
        mode = statistics.mode(numbers)
    except statistics.StatisticsError:
        mode = "No mode"

    print(f"Mean: {mean}")
    print(f"Median: {median}")
    print(f"Mode: {mode}")
calculate_mean_median_mode()    
\end{lstlisting}

\end{document}