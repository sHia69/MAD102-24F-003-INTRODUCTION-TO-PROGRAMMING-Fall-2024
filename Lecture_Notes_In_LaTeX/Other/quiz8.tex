\documentclass{article}
\usepackage{amsmath}

\begin{document}

\section*{Python Questions and Answers}

\subsection*{Question 1}
What will be the output of the following Python code?
\begin{verbatim}
a = [1, 2, 3, 4]
b = [sum(a[0:x+1]) for x in range(0, len(a))]
print(b)
\end{verbatim}

\textbf{Answer:} \([1, 3, 6, 10]\)

\subsection*{Question 2}
This method removes and returns the last item in a list.
\begin{itemize}
    \item list.last()
    \item list.removeLast()
    \item list.pop()
    \item list.popLast()
\end{itemize}

\textbf{Answer:} \texttt{list.pop()}

\subsection*{Question 3}
This method sorts a list and returns a new list instead of modifying it in place.
\begin{itemize}
    \item sortedBy()
    \item sorted()
    \item newSort()
    \item sort()
\end{itemize}

\textbf{Answer:} \texttt{sorted()}

\subsection*{Question 4}
\begin{verbatim}
names = ['Tony', 'Bruce', 'Natasha', 'Steve']
print(names[3:])
\end{verbatim}

\begin{itemize}
    \item Steve
    \item Error - missing a value.
    \item Natasha, Steve
    \item Tony, Bruce, Natasha
\end{itemize}

\textbf{Answer:} \texttt{['Steve']}

\subsection*{Question 5}
The count method for a list returns the number of unique items in the list - it does not count duplicates.
\begin{itemize}
    \item True
    \item False
\end{itemize}

\textbf{Answer:} False

\subsection*{Question 6}
Which one of these best describes a list:
\begin{itemize}
    \item An immutable container that contains a sequence of items ordered from right-to-left
    \item A mutable container that contains a sequence of items ordered from left-to-right
    \item An immutable container that contains items with no defined order
    \item A mutable container that contains items with no defined order
\end{itemize}

\textbf{Answer:} A mutable container that contains a sequence of items ordered from left-to-right

\subsection*{Question 7}
\texttt{all(list)} - this method returns True if every element in list is True (!=0), or if the list is empty.
\begin{itemize}
    \item True
    \item False
\end{itemize}

\textbf{Answer:} True

\subsection*{Question 8}
What will be the output of the following Python code?
\begin{verbatim}
lst = [3, 4, 6, 1, 2]
lst[1:2] = [7, 8]
print(lst)
\end{verbatim}

\begin{itemize}
    \item [3, 7, 6, 1, 2]
    \item [3, [7, 8], 6, 1, 2]
    \item Syntax Error
    \item [3, 7, 8, 6, 1, 2]
\end{itemize}

\textbf{Answer:} \([3, 7, 8, 6, 1, 2]\)

\subsection*{Question 9}
If I applied the title method to the string - 'This Is a Title', the resulting string returned would be
\begin{itemize}
    \item This Is A Title
    \item THIS IS A TITLE
    \item this is a title
    \item This is a title
\end{itemize}

\textbf{Answer:} This Is A Title

\subsection*{Question 10}
What will be the output of the following Python code?
\begin{verbatim}
numbers = [1, 2, 3, 4]
numbers.append([5, 6, 7, 8])
print(len(numbers))
\end{verbatim}

\begin{itemize}
    \item 5
    \item 6
    \item 7
    \item 8
\end{itemize}

\textbf{Answer:} 5

\end{document}