\documentclass{article}
\usepackage{xcolor}
\usepackage{url}
\usepackage{hyperref}
\usepackage{graphicx}
\usepackage{soul}
\usepackage{enumerate}
\usepackage{fancyhdr}
\usepackage{listings}

% Define custom colors
\definecolor{bg}{rgb}{0.64, 0.64, 0.82}
\definecolor{frame}{rgb}{0.59, 0.47, 0.71}
\definecolor{keyword}{rgb}{0.63, 0.36, 0.94}
\definecolor{comment}{rgb}{0.44, 0.5, 0.56}
\definecolor{string}{rgb}{0.56, 0.27, 0.52}

% Set custom listings options
\lstset{
    backgroundcolor=\color{bg},
    frame=single,
    rulecolor=\color{frame},
    basicstyle=\ttfamily\small,
    keywordstyle=\color{keyword}\bfseries,
    commentstyle=\color{comment},
    stringstyle=\color{string},
    showstringspaces=false,
    breaklines=true,
    xleftmargin=2mm,
    xrightmargin=2mm
}
% Header
\pagestyle{fancy}
\fancyhf{}
\fancyhead[L]{MAD 102 - Intro to Prog - Fall 2024}
\fancyhead[R]{Instructor: Aishwarya Rajasekaran \thepage}

\title{Types - MAD 102 Week 2 Notes}
\author{Hia Al Saleh}
\date{September 11th, 2024}

\begin{document}

\maketitle
\tableofcontents
\newpage 

\section{Getting Input}
\begin{itemize}
    \item Information required for the program to operate comes as input.
    \item Use \texttt{input()} function to capture input as text (string).
    \item Example:
\begin{lstlisting}[language=python]
name = input("Enter your name")
print("Hello", name)
    \end{lstlisting}
\end{itemize}

\subsection{Input Details}
\begin{itemize}
    \item The input string can be assigned to a variable.
    \item The argument in \texttt{input()} represents prompt text displayed to the user.
    \item Input always returns a string, so numerical input needs conversion.
\end{itemize}

\subsection{Converting Input Types}
\textbf{Explicit Conversion}:
\begin{lstlisting}[language=python]
age = int(input("Enter your Age"))
print(type(age))  # Data Type: <class 'int'>
\end{lstlisting}

\textbf{Implicit Conversion}:
\begin{lstlisting}[language=python]
num1 = 1
num2 = 3.4
sum = num1 + num2  # Data Type: <class 'float'>
\end{lstlisting}

\section{Outputting Information}
\begin{itemize}
    \item Use the \texttt{print()} function to display results in the console.
    \item Multiple arguments in \texttt{print()} are separated by commas and displayed with spaces in between.
    \item \texttt{print()} ends with a newline character by default.
    \item You can use \texttt{end=""} to avoid moving to the next line.
\end{itemize}
Example:
\begin{lstlisting}[language=python]
print("Hello", end='')
print(" World!")  # Output: Hello World!
\end{lstlisting}

\section{Strings}
\begin{itemize}
    \item Strings are a sequence of characters enclosed in single or double quotes.
    \item Strings are immutable (cannot be changed once created).
    \item Access individual characters using indexing:
\begin{lstlisting}[language=python]
name = 'Luke'
print(name[1])  # Output: u
\end{lstlisting}
    \item Positive indices start from 0 (left to right); negative indices start from -1 (right to left).
    \item Strings can be concatenated using the \texttt{+} operator.
\end{itemize}

\subsection{Formatted Strings}
\begin{itemize}
    \item Use f-strings with \texttt{f" "} to format strings with placeholders \{\}.
    \item Example:
\begin{lstlisting}[language=python]
name = "John"
age = 25
print(f"{name} is {age} years old.")
    \end{lstlisting}
    \item Formatting options can be applied inside placeholders using \texttt{':'}. For example, formatting numbers.
\end{itemize}

\section{Lists}
\begin{itemize}
    \item A list is a mutable, ordered collection of elements, defined with square brackets \texttt{[ ]}.
    \item Access list elements using their index:
\begin{lstlisting}[language=python]
my_list = [1, 2, 3]
print(my_list[0])  # Output: 1
\end{lstlisting}
    \item Lists can be modified, and new items can be added using the \texttt{append()} method.
    \item Items can be removed using \texttt{pop()} or \texttt{remove()} methods.
\end{itemize}

\section{Tuples}
\begin{itemize}
    \item Tuples are immutable, ordered collections of elements, defined using parentheses \texttt{( )}.
    \item Example:
    \begin{lstlisting}[language=python]
coordinates = (83.232, 32.321)
    \end{lstlisting}
    \item Named tuples allow attributes to be accessed using dot notation.
\end{itemize}

\section{Sets}
\begin{itemize}
    \item Sets are unordered collections of unique elements, defined using curly braces \texttt{\{ \}}.
    \item No repeated elements are allowed.
    \item Add elements using the \texttt{add()} method and remove elements using \texttt{remove()} or \texttt{pop()}.
\end{itemize}

\section{Dictionaries}
\begin{itemize}
    \item Dictionaries store key-value pairs, defined using curly braces \texttt{\{ \}} with a colon separating keys and values.
    \item Access items using keys rather than indices.
    \item Modify dictionaries by assigning new values to keys, and remove entries using the \texttt{del} keyword.
\end{itemize}
\begin{lstlisting}[language=python]
    nhteams = {
    1926: 'Detroit Red Wings',
    1979: 'Edmonton Oilers',
    1927: 'Toronto Maple Leafs'
    }
    nhteams[1926]= 'Chicago Blackhawks'

    print(nhteams)
\end{lstlisting}

\end{document}
